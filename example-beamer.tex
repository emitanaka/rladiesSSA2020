% Options for packages loaded elsewhere
\PassOptionsToPackage{unicode}{hyperref}
\PassOptionsToPackage{hyphens}{url}
%
\documentclass[
  10pt,
  ignorenonframetext,
  aspectratio=169,
]{beamer}
\usepackage{pgfpages}
\setbeamertemplate{caption}[numbered]
\setbeamertemplate{caption label separator}{: }
\setbeamercolor{caption name}{fg=normal text.fg}
\beamertemplatenavigationsymbolsempty
% Prevent slide breaks in the middle of a paragraph
\widowpenalties 1 10000
\raggedbottom
\setbeamertemplate{part page}{
  \centering
  \begin{beamercolorbox}[sep=16pt,center]{part title}
    \usebeamerfont{part title}\insertpart\par
  \end{beamercolorbox}
}
\setbeamertemplate{section page}{
  \centering
  \begin{beamercolorbox}[sep=12pt,center]{part title}
    \usebeamerfont{section title}\insertsection\par
  \end{beamercolorbox}
}
\setbeamertemplate{subsection page}{
  \centering
  \begin{beamercolorbox}[sep=8pt,center]{part title}
    \usebeamerfont{subsection title}\insertsubsection\par
  \end{beamercolorbox}
}
\AtBeginPart{
  \frame{\partpage}
}
\AtBeginSection{
  \ifbibliography
  \else
    \frame{\sectionpage}
  \fi
}
\AtBeginSubsection{
  \frame{\subsectionpage}
}
\usepackage{lmodern}
\usepackage{amssymb,amsmath}
\usepackage{ifxetex,ifluatex}
\ifnum 0\ifxetex 1\fi\ifluatex 1\fi=0 % if pdftex
  \usepackage[T1]{fontenc}
  \usepackage[utf8]{inputenc}
  \usepackage{textcomp} % provide euro and other symbols
\else % if luatex or xetex
  \ifxetex
    \usepackage{mathspec}
  \else
    \usepackage{unicode-math}
  \fi
  \defaultfontfeatures{Scale=MatchLowercase}
  \defaultfontfeatures[\rmfamily]{Ligatures=TeX,Scale=1}
\fi
\usetheme[]{Copenhagen}
\usecolortheme{dolphin}
\usefonttheme{structurebold}
% Use upquote if available, for straight quotes in verbatim environments
\IfFileExists{upquote.sty}{\usepackage{upquote}}{}
\IfFileExists{microtype.sty}{% use microtype if available
  \usepackage[]{microtype}
  \UseMicrotypeSet[protrusion]{basicmath} % disable protrusion for tt fonts
}{}
\makeatletter
\@ifundefined{KOMAClassName}{% if non-KOMA class
  \IfFileExists{parskip.sty}{%
    \usepackage{parskip}
  }{% else
    \setlength{\parindent}{0pt}
    \setlength{\parskip}{6pt plus 2pt minus 1pt}}
}{% if KOMA class
  \KOMAoptions{parskip=half}}
\makeatother
\usepackage{xcolor}
\IfFileExists{xurl.sty}{\usepackage{xurl}}{} % add URL line breaks if available
\IfFileExists{bookmark.sty}{\usepackage{bookmark}}{\usepackage{hyperref}}
\hypersetup{
  pdftitle={This is a Beamer presentation made using R Markdown},
  pdfauthor={Emi Tanaka},
  hidelinks,
  pdfcreator={LaTeX via pandoc}}
\urlstyle{same} % disable monospaced font for URLs
\newif\ifbibliography
\setlength{\emergencystretch}{3em} % prevent overfull lines
\providecommand{\tightlist}{%
  \setlength{\itemsep}{0pt}\setlength{\parskip}{0pt}}
\setcounter{secnumdepth}{-\maxdimen} % remove section numbering
\usepackage{fancyvrb}

\title{This is a Beamer presentation made using R Markdown}
\subtitle{\texttt{rmarkdown::beamer\_presentation}}
\author{Emi Tanaka}
\date{30th June 2020}
\institute{Monash University}

\begin{document}
\frame{\titlepage}

\begin{frame}
  \tableofcontents[hideallsubsections]
\end{frame}
\hypertarget{basics}{%
\section{Basics}\label{basics}}

\begin{frame}[fragile]{Setting up the YAML}
\protect\hypertarget{setting-up-the-yaml}{}

\begin{block}{R Markdown YAML}

\texttt{output:\ beamer\_presentation}

\end{block}

Also check out the
\href{https://github.com/eddelbuettel/binb}{\texttt{binb} R-package}.

\end{frame}

\hypertarget{clean-syntax}{%
\section{Clean syntax}\label{clean-syntax}}

\begin{frame}{Example: Writing mathematics is as usual}
\protect\hypertarget{example-writing-mathematics-is-as-usual}{}

Consider a general linear mixed models:

\[\boldsymbol{y} = \mathbf{X}\boldsymbol{\beta} + \mathbf{Z}\boldsymbol{u} + \boldsymbol{e}\]
where

\begin{itemize}
\tightlist
\item
  \(\boldsymbol{y}\) is a \(n\times 1\) vector of observations,
\item
  \(\mathbf{X}\) is a \(n\times p\) design matrix for fixed effects
  \(\boldsymbol{\beta}\),
\item
  \(\mathbf{Z}\) is a \(n\times q\) design matrix for fixed effects
  \(\boldsymbol{u}\), and
\item
  \(\boldsymbol{e}\) is a \(n\times 1\) vector of random error.
\end{itemize}

\end{frame}

\begin{frame}[fragile]{Simplified content writing: itemized list}
\protect\hypertarget{simplified-content-writing-itemized-list}{}

\begin{columns}[T]
\begin{column}{0.48\textwidth}
\begin{block}{Plain \LaTeX}

\small

\begin{verbatim}
\begin{itemize}
\item 
  $\boldsymbol{y}$ is a 
  $n\times 1$ vector of 
  observations,
\item
  $\mathbf{X}$ is a $n\times p$ 
  design matrix for fixed effects
  $\boldsymbol{\beta}$$,
\item
  $\mathbf{Z}$ is a $n\times q$ 
  design matrix for fixed effects
  $\boldsymbol{u}$, and
\item
  $\boldsymbol{e}$ is a $n\times 1$ 
  vector of random error.
\end{itemize}
\end{verbatim}

\end{block}
\end{column}

\begin{column}{0.48\textwidth}
\begin{block}{Markdown syntax}

\small

\vspace{0.5cm}

\begin{verbatim}
* $\boldsymbol{y}$ is a $n\times 1$ 
  vector of observations,
  
* $\mathbf{X}$ is a $n\times p$ 
  design matrix for fixed effects 
  $\boldsymbol{\beta}$,
  
* $\mathbf{Z}$ is a $n\times q$ 
  design matrix for fixed effects 
  $\boldsymbol{u}$, and
  
* $\boldsymbol{e}$ is a $n\times 1$ 
  vector of random error.
\end{verbatim}

\end{block}
\end{column}
\end{columns}

\end{frame}

\begin{frame}[fragile]{Make new frame}
\protect\hypertarget{make-new-frame}{}

\begin{columns}[T]
\begin{column}{0.48\textwidth}
Instead of

\begin{block}{\LaTeX}

\begin{verbatim}
\begin{frame}{Title}
Content
\ end{frame}
\end{verbatim}

\end{block}

do

\begin{block}{markdown}

\begin{verbatim}
## Title
Content
\end{verbatim}

\end{block}
\end{column}

\begin{column}{0.48\textwidth}
In the \textbf{R Markdown YAML} set:

\begin{verbatim}
output:
  beamer_presentation:
    slide_level: 2
\end{verbatim}

\begin{itemize}
\tightlist
\item
  then a heading at slide level starts a new frame; or
\item
  a horizontal line, like \texttt{-\/-\/-}, starts a new frame.
\end{itemize}
\end{column}
\end{columns}

\end{frame}

\begin{frame}[fragile]{Multi-column output}
\protect\hypertarget{multi-column-output}{}

Stick with \LaTeX or Pandoc syntax (both will work)

\begin{columns}[T]
\begin{column}{0.48\textwidth}
\begin{block}{Plain \LaTeX}

\begin{verbatim}
\begin{columns}
\begin{column}{0.5\textwidth}
Column 1 content
\end{column}
\begin{column}{0.5\textwidth}
Column 2 content
\end{column}
\end{columns}
\end{verbatim}

\end{block}
\end{column}

\begin{column}{0.48\textwidth}
\begin{block}{Pandoc syntax}

\begin{verbatim}
:::: columns
::: column
Column 1 content
:::
::: column
Column 2 content
:::
::::
\end{verbatim}

\end{block}
\end{column}
\end{columns}

\end{frame}

\hypertarget{appearance}{%
\section{Appearance}\label{appearance}}

\begin{frame}[fragile]{Pandoc options}
\protect\hypertarget{pandoc-options}{}

\begin{itemize}
\tightlist
\item
  See
  \href{https://pandoc.org/MANUAL.html\#variables-for-beamer-slides}{Pandoc
  manual} for all the variables available to modify in the YAML.

  \begin{itemize}
  \tightlist
  \item
    E.g. \texttt{aspectratio:\ 43} for 4:3 ratio.
  \end{itemize}
\item
  See also
  \href{https://pandoc.org/MANUAL.html\#producing-slide-shows-with-pandoc}{Producing
  slide shows with pandoc}

  \begin{itemize}
  \tightlist
  \item
    E.g. Speaker notes
  \end{itemize}
\end{itemize}

\end{frame}

\begin{frame}[fragile,shrink]{Beamer frame atributes}
\protect\hypertarget{beamer-frame-atributes}{}

\begin{verbatim}
## Beamer frame atributes {.shrink}
\end{verbatim}

\begin{itemize}
\tightlist
\item
  The frame attributes in Section 8.1 of the
  \href{http://mirror.aarnet.edu.au/pub/CTAN/macros/latex/contrib/beamer/doc/beameruserguide.pdf}{Beamer
  User's Guide} can be used if inserted after header level as
  \texttt{\{.attribute\}} where \texttt{attribute} replaced with

  \begin{itemize}
  \tightlist
  \item
    \texttt{allowdisplaybreaks}
  \item
    \texttt{allowframebreaks}
  \item
    \texttt{b}
  \item
    \texttt{c}
  \item
    \texttt{t}
  \item
    \texttt{environment}
  \item
    \texttt{label}
  \item
    \texttt{plain}
  \item
    \texttt{shrink}
  \item
    \texttt{standout}
  \item
    \texttt{noframenumbering}
  \end{itemize}
\end{itemize}

\end{frame}

\hypertarget{animation}{%
\section{Animation}\label{animation}}

\begin{frame}[fragile]{Animation}

\begin{itemize}[<+->]
\tightlist
\item
  First step
\item
  Second step
\item
  Third step
\end{itemize}

\pause

You can also include pauses with \texttt{.\ .\ .}

\end{frame}

\end{document}
