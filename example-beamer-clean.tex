\documentclass[10pt,aspectratio=169]{beamer}
\usepackage{amssymb,amsmath}
\usepackage{lmodern}
\usepackage{hyperref, url}
\usetheme{Copenhagen}
\usecolortheme{dolphin}
\usefonttheme{structurebold}


\title{This is a Beamer presentation made using \LaTeX}
\subtitle{}
\author{Emi Tanaka}
\date{30th June 2020}
\institute{Monash University}

\begin{document}
\frame{\titlepage}

\begin{frame}
  \tableofcontents[hideallsubsections]
\end{frame}

\section{Basics}

\begin{frame}{Setting up the YAML}

\begin{block}{R Markdown YAML}

\texttt{output:\ beamer\_presentation}

\end{block}

Also check out the
\href{https://github.com/eddelbuettel/binb}{\texttt{binb} R-package}.

\end{frame}

\section{Clean syntax}

\begin{frame}{Example: Writing mathematics is as usual}

Consider a general linear mixed models:

$\boldsymbol{y} = \mathbf{X}\boldsymbol{\beta} + \mathbf{Z}\boldsymbol{u} + \boldsymbol{e}$
where

\begin{itemize}
\item
  $\boldsymbol{y}$ is a $n\times 1$ vector of observations,
\item
  $\mathbf{X}$ is a $n\times p$ design matrix for fixed effects
  $\boldsymbol{\beta}$,
\item
  $\mathbf{Z}$ is a $n\times q$ design matrix for fixed effects
  $\boldsymbol{u}$, and
\item
  $\boldsymbol{e}$ is a $n\times 1$ vector of random error.
\end{itemize}

\end{frame}

\begin{frame}[fragile]{Simplified content writing: itemized list}

\begin{columns}[T]
\begin{column}{0.48\textwidth}
\begin{block}{Plain \LaTeX}

\small

\begin{verbatim}
\begin{itemize}
\item 
  $\boldsymbol{y}$ is a 
  $n\times 1$ vector of 
  observations,
\item
  $\mathbf{X}$ is a $n\times p$ 
  design matrix for fixed effects
  $\boldsymbol{\beta}$$,
\item
  $\mathbf{Z}$ is a $n\times q$ 
  design matrix for fixed effects
  $\boldsymbol{u}$, and
\item
  $\boldsymbol{e}$ is a $n\times 1$ 
  vector of random error.
\end{itemize}
\end{verbatim}

\end{block}
\end{column}

\begin{column}{0.48\textwidth}
\begin{block}{Markdown syntax}

\small

\vspace{0.5cm}

\begin{verbatim}
* $\boldsymbol{y}$ is a $n\times 1$ 
  vector of observations,
  
* $\mathbf{X}$ is a $n\times p$ 
  design matrix for fixed effects 
  $\boldsymbol{\beta}$,
  
* $\mathbf{Z}$ is a $n\times q$ 
  design matrix for fixed effects 
  $\boldsymbol{u}$, and
  
* $\boldsymbol{e}$ is a $n\times 1$ 
  vector of random error.
\end{verbatim}

\end{block}
\end{column}
\end{columns}

\end{frame}

\begin{frame}[fragile]{Make new frame}

\begin{columns}[T]
\begin{column}{0.48\textwidth}
Instead of

\begin{block}{\LaTeX}

\begin{verbatim}
\begin{frame}{Title}
Content
\ end{frame}
\end{verbatim}

\end{block}

do

\begin{block}{markdown}

\begin{verbatim}
## Title
Content
\end{verbatim}

\end{block}
\end{column}

\begin{column}{0.48\textwidth}
In the \textbf{R Markdown YAML} set:

\begin{verbatim}
output:
  beamer_presentation:
    slide_level: 2
\end{verbatim}

\begin{itemize}
\item
  then a heading at slide level starts a new frame; or
\item
  a horizontal line, like \texttt{-\/-\/-}, starts a new frame.
\end{itemize}
\end{column}
\end{columns}

\end{frame}

\begin{frame}[fragile]{Multi-column output}

Stick with \LaTeX or Pandoc syntax (both will work)

\begin{columns}[T]
\begin{column}{0.48\textwidth}
\begin{block}{Plain \LaTeX}

\begin{verbatim}
\begin{columns}
\begin{column}{0.5\textwidth}
Column 1 content
\end{column}
\begin{column}{0.5\textwidth}
Column 2 content
\end{column}
\end{columns}
\end{verbatim}

\end{block}
\end{column}

\begin{column}{0.48\textwidth}
\begin{block}{Pandoc syntax}

\begin{verbatim}
:::: columns
::: column
Column 1 content
:::
::: column
Column 2 content
:::
::::
\end{verbatim}

\end{block}
\end{column}
\end{columns}

\end{frame}


\section{Appearance}

\begin{frame}[fragile]{Pandoc options}

\begin{itemize}
\item
  See
  \href{https://pandoc.org/MANUAL.html\#variables-for-beamer-slides}{Pandoc
  manual} for all the variables available to modify in the YAML.

  \begin{itemize}
  \item
    E.g. \texttt{aspectratio:\ 43} for 4:3 ratio.
  \end{itemize}
\item
  See also
  \href{https://pandoc.org/MANUAL.html\#producing-slide-shows-with-pandoc}{Producing
  slide shows with pandoc}

  \begin{itemize}
  \item
    E.g. Speaker notes
  \end{itemize}
\end{itemize}

\end{frame}

\begin{frame}[fragile,shrink]{Beamer frame atributes}

\begin{verbatim}
## Beamer frame atributes {.shrink}
\end{verbatim}

\begin{itemize}
\item
  The frame attributes in Section 8.1 of the
  \href{http://mirror.aarnet.edu.au/pub/CTAN/macros/latex/contrib/beamer/doc/beameruserguide.pdf}{Beamer
  User's Guide} can be used if inserted after header level as
  \texttt{\{.attribute\}} where \texttt{attribute} replaced with

  \begin{itemize}
  \item
    \texttt{allowdisplaybreaks}
  \item
    \texttt{allowframebreaks}
  \item
    \texttt{b}
  \item
    \texttt{c}
  \item
    \texttt{t}
  \item
    \texttt{environment}
  \item
    \texttt{label}
  \item
    \texttt{plain}
  \item
    \texttt{shrink}
  \item
    \texttt{standout}
  \item
    \texttt{noframenumbering}
  \end{itemize}
\end{itemize}

\end{frame}


\section{Animation}

\begin{frame}[fragile]{Animation}

\begin{itemize}[<+->]
\item
  First step
\item
  Second step
\item
  Third step
\end{itemize}

\pause

You can also include pauses with \texttt{.\ .\ .}

\end{frame}

\end{document}
